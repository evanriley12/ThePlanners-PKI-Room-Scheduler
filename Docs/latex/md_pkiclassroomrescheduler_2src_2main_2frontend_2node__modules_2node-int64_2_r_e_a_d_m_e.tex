\chapter{README}
\hypertarget{md_pkiclassroomrescheduler_2src_2main_2frontend_2node__modules_2node-int64_2_r_e_a_d_m_e}{}\label{md_pkiclassroomrescheduler_2src_2main_2frontend_2node__modules_2node-int64_2_r_e_a_d_m_e}\index{README@{README}}
Java\+Script Numbers are represented as \href{http://steve.hollasch.net/cgindex/coding/ieeefloat.html}{\texttt{ IEEE 754 double-\/precision floats}}. Unfortunately, this means they lose integer precision for values beyond +/-\/ 2\texorpdfstring{$^\wedge$}{\string^}\texorpdfstring{$^\wedge$}{\string^}53. For projects that need to accurately handle 64-\/bit ints, such as \href{https://github.com/wadey/node-thrift}{\texttt{ node-\/thrift}}, a performant, Number-\/like class is needed. Int64 is that class.

Int64 instances look and feel much like JS-\/native Numbers. By way of example ... 
\begin{DoxyCode}{0}
\DoxyCodeLine{//\ First,\ let's\ illustrate\ the\ problem\ ...}
\DoxyCodeLine{>\ (0x123456789).toString(16)}
\DoxyCodeLine{'123456789'\ //\ <-\/\ what\ we\ expect.}
\DoxyCodeLine{>\ (0x123456789abcdef0).toString(16)}
\DoxyCodeLine{'123456789abcdf00'\ //\ <-\/\ Ugh!\ \ JS\ doesn't\ do\ big\ ints.\ :(}
\DoxyCodeLine{}
\DoxyCodeLine{//\ So\ let's\ create\ a\ couple\ Int64s\ using\ the\ above\ values\ ...}
\DoxyCodeLine{}
\DoxyCodeLine{//\ Require,\ of\ course}
\DoxyCodeLine{>\ Int64\ =\ require('node-\/int64')}
\DoxyCodeLine{}
\DoxyCodeLine{//\ x's\ value\ is\ what\ we\ expect\ (the\ decimal\ value\ of\ 0x123456789)}
\DoxyCodeLine{>\ x\ =\ new\ Int64(0x123456789)}
\DoxyCodeLine{[Int64\ value:4886718345\ octets:00\ 00\ 00\ 01\ 23\ 45\ 67\ 89]}
\DoxyCodeLine{}
\DoxyCodeLine{//\ y's\ value\ is\ Infinity\ because\ it's\ outside\ the\ range\ of\ integer}
\DoxyCodeLine{//\ precision.\ \ But\ that's\ okay\ -\/\ it's\ still\ useful\ because\ it's\ internal}
\DoxyCodeLine{//\ representation\ (octets)\ is\ what\ we\ passed\ in}
\DoxyCodeLine{>\ y\ =\ new\ Int64('123456789abcdef0')}
\DoxyCodeLine{[Int64\ value:Infinity\ octets:12\ 34\ 56\ 78\ 9a\ bc\ de\ f0]}
\DoxyCodeLine{}
\DoxyCodeLine{//\ Let's\ do\ some\ math.\ \ Int64's\ behave\ like\ Numbers.\ \ (Sorry,\ Int64\ isn't}
\DoxyCodeLine{//\ for\ doing\ 64-\/bit\ integer\ arithmetic\ (yet)\ -\/\ it's\ just\ for\ carrying}
\DoxyCodeLine{//\ around\ int64\ values}
\DoxyCodeLine{>\ x\ +\ 1}
\DoxyCodeLine{4886718346}
\DoxyCodeLine{>\ y\ +\ 1}
\DoxyCodeLine{Infinity}
\DoxyCodeLine{}
\DoxyCodeLine{//\ Int64\ string\ operations\ ...}
\DoxyCodeLine{>\ 'value:\ '\ +\ x}
\DoxyCodeLine{'value:\ 4886718345'}
\DoxyCodeLine{>\ 'value:\ '\ +\ y}
\DoxyCodeLine{'value:\ Infinity'}
\DoxyCodeLine{>\ x.toString(2)}
\DoxyCodeLine{'100100011010001010110011110001001'}
\DoxyCodeLine{>\ y.toString(2)}
\DoxyCodeLine{'Infinity'}
\DoxyCodeLine{}
\DoxyCodeLine{//\ Use\ JS's\ isFinite()\ method\ to\ see\ if\ the\ Int64\ value\ is\ in\ the}
\DoxyCodeLine{//\ integer-\/precise\ range\ of\ JS\ values}
\DoxyCodeLine{>\ isFinite(x)}
\DoxyCodeLine{true}
\DoxyCodeLine{>\ isFinite(y)}
\DoxyCodeLine{false}
\DoxyCodeLine{}
\DoxyCodeLine{//\ Get\ an\ octet\ string\ representation.\ \ (Yay,\ y\ is\ what\ we\ put\ in!)}
\DoxyCodeLine{>\ x.toOctetString()}
\DoxyCodeLine{'0000000123456789'}
\DoxyCodeLine{>\ y.toOctetString()}
\DoxyCodeLine{'123456789abcdef0'}
\DoxyCodeLine{}
\DoxyCodeLine{//\ Finally,\ some\ other\ ways\ to\ create\ Int64s\ ...}
\DoxyCodeLine{}
\DoxyCodeLine{//\ Pass\ hi/lo\ words}
\DoxyCodeLine{>\ new\ Int64(0x12345678,\ 0x9abcdef0)}
\DoxyCodeLine{[Int64\ value:Infinity\ octets:12\ 34\ 56\ 78\ 9a\ bc\ de\ f0]}
\DoxyCodeLine{}
\DoxyCodeLine{//\ Pass\ a\ Buffer}
\DoxyCodeLine{>\ new\ Int64(new\ Buffer([0x12,\ 0x34,\ 0x56,\ 0x78,\ 0x9a,\ 0xbc,\ 0xde,\ 0xf0]))}
\DoxyCodeLine{[Int64\ value:Infinity\ octets:12\ 34\ 56\ 78\ 9a\ bc\ de\ f0]}
\DoxyCodeLine{}
\DoxyCodeLine{//\ Pass\ a\ Buffer\ and\ offset}
\DoxyCodeLine{>\ new\ Int64(new\ Buffer([0,0,0,0,0x12,\ 0x34,\ 0x56,\ 0x78,\ 0x9a,\ 0xbc,\ 0xde,\ 0xf0]),\ 4)}
\DoxyCodeLine{[Int64\ value:Infinity\ octets:12\ 34\ 56\ 78\ 9a\ bc\ de\ f0]}
\DoxyCodeLine{}
\DoxyCodeLine{//\ Pull\ out\ into\ a\ buffer}
\DoxyCodeLine{>\ new\ Int64(new\ Buffer([0x12,\ 0x34,\ 0x56,\ 0x78,\ 0x9a,\ 0xbc,\ 0xde,\ 0xf0])).toBuffer()}
\DoxyCodeLine{<Buffer\ 12\ 34\ 56\ 78\ 9a\ bc\ de\ f0>}
\DoxyCodeLine{}
\DoxyCodeLine{//\ Or\ copy\ into\ an\ existing\ one\ (at\ an\ offset)}
\DoxyCodeLine{>\ var\ buf\ =\ new\ Buffer(1024);}
\DoxyCodeLine{>\ new\ Int64(new\ Buffer([0x12,\ 0x34,\ 0x56,\ 0x78,\ 0x9a,\ 0xbc,\ 0xde,\ 0xf0])).copy(buf,\ 512);}

\end{DoxyCode}
 