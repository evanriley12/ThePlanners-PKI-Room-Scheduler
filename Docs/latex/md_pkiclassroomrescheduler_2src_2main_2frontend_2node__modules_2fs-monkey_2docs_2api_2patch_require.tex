\chapter{\texorpdfstring{$<$}{<}tt\texorpdfstring{$>$}{>}patch\+Require(vol\mbox{[}, unixify\+Paths\mbox{[}, Module\mbox{]}\mbox{]})\texorpdfstring{$<$}{<}/tt\texorpdfstring{$>$}{>}}
\hypertarget{md_pkiclassroomrescheduler_2src_2main_2frontend_2node__modules_2fs-monkey_2docs_2api_2patch_require}{}\label{md_pkiclassroomrescheduler_2src_2main_2frontend_2node__modules_2fs-monkey_2docs_2api_2patch_require}\index{$<$tt$>$patchRequire(vol\mbox{[}, unixifyPaths\mbox{[}, Module\mbox{]}\mbox{]})$<$/tt$>$@{$<$tt$>$patchRequire(vol[, unixifyPaths[, Module]])$<$/tt$>$}}
\label{md_pkiclassroomrescheduler_2src_2main_2frontend_2node__modules_2fs-monkey_2docs_2api_2patch_require_autotoc_md13085}%
\Hypertarget{md_pkiclassroomrescheduler_2src_2main_2frontend_2node__modules_2fs-monkey_2docs_2api_2patch_require_autotoc_md13085}%
 Patches Node\textquotesingle{}s {\ttfamily module} module to use a given {\itshape fs-\/like} object {\ttfamily vol} for module loading.


\begin{DoxyItemize}
\item {\ttfamily vol} -\/ fs-\/like object
\item {\ttfamily unixify\+Paths} \texorpdfstring{$\ast$}{*}(optional)\texorpdfstring{$\ast$}{*} -\/ whether to convert Windows paths to unix style paths, defaults to {\ttfamily false}.
\item {\ttfamily Module} \texorpdfstring{$\ast$}{*}(optional)\texorpdfstring{$\ast$}{*} -\/ a module to patch, defaults to `require(\textquotesingle{}module')\`{}
\end{DoxyItemize}

Monkey-\/patches the {\ttfamily require} function in Node, this way you can make Node.\+js to {\itshape require} modules from your custom filesystem.

It expects an object with three filesystem methods implemented that are needed for the {\ttfamily require} function to work.


\begin{DoxyCode}{0}
\DoxyCodeLine{let\ vol\ =\ \{}
\DoxyCodeLine{\ \ \ \ readFileSync:\ ()\ =>\ \{\},}
\DoxyCodeLine{\ \ \ \ realpathSync:\ ()\ =>\ \{\},}
\DoxyCodeLine{\ \ \ \ statSync:\ ()\ =>\ \{\},}
\DoxyCodeLine{\};}

\end{DoxyCode}


If you want to make Node.\+js to {\itshape require} your files from memory, you don\textquotesingle{}t need to implement those functions yourself, just use the \href{https://github.com/streamich/memfs}{\texttt{ {\ttfamily memfs}}} package\+:


\begin{DoxyCode}{0}
\DoxyCodeLine{import\ \{vol\}\ from\ 'memfs';}
\DoxyCodeLine{import\ \{patchRequire\}\ from\ 'fs-\/monkey';}
\DoxyCodeLine{}
\DoxyCodeLine{vol.fromJSON(\{'/foo/bar.js':\ 'console.log("{}obi\ trice"{});'\});}
\DoxyCodeLine{patchRequire(vol);}
\DoxyCodeLine{require('/foo/bar');\ //\ obi\ trice}

\end{DoxyCode}


Now the {\ttfamily require} function will only load the files from the {\ttfamily vol} file system, but not from the actual filesystem on the disk.

If you want the {\ttfamily require} function to load modules from both file systems, use the \href{https://github.com/streamich/unionfs}{\texttt{ {\ttfamily unionfs}}} package to combine both filesystems into a union\+:


\begin{DoxyCode}{0}
\DoxyCodeLine{import\ \{vol\}\ from\ 'memfs';}
\DoxyCodeLine{import\ \{patchRequire\}\ from\ 'fs-\/monkey';}
\DoxyCodeLine{import\ \{ufs\}\ from\ 'unionfs';}
\DoxyCodeLine{import\ *\ as\ fs\ from\ 'fs';}
\DoxyCodeLine{}
\DoxyCodeLine{vol.fromJSON(\{'/foo/bar.js':\ 'console.log("{}obi\ trice"{});'\});}
\DoxyCodeLine{ufs}
\DoxyCodeLine{\ \ \ \ .use(vol)}
\DoxyCodeLine{\ \ \ \ .use(fs);}
\DoxyCodeLine{patchRequire(ufs);}
\DoxyCodeLine{require('/foo/bar.js');\ //\ obi\ trice}

\end{DoxyCode}
 